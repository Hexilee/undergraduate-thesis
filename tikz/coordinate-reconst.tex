\documentclass[tikz]{standalone}

% \documentclass{article}

% \usetikzlibrary{calc,3d,arrows}% tikz package already loaded by 'tikz' option
\usepackage{tikz-3dplot}
\usepackage{tikz}   %TikZ is required for this to work.  Make sure this exists before the next line

\usepackage[active,tightpage]{preview}  %generates a tightly fitting border around the work

\PreviewEnvironment{tikzpicture}
\setlength\PreviewBorder{2mm}

\begin{document}

%Angle Definitions
%-----------------

%set the plot display orientation
%synatax: \tdplotsetdisplay{\theta_d}{\phi_d}
\tdplotsetmaincoords{60}{110}

%define polar coordinates for some vector
%TODO: look into using 3d spherical coordinate system
\pgfmathsetmacro{\rvec}{.41}
\pgfmathsetmacro{\betavec}{15}
\pgfmathsetmacro{\thetavec}{30}
\pgfmathsetmacro{\phivec}{60}

%start tikz picture, and use the tdplot_main_coords style to implement the display 
%coordinate transformation provided by 3dplot
\begin{tikzpicture}[scale=5,tdplot_main_coords]

%set up some coordinates 
%-----------------------
\coordinate (O) at (0,0,0) node[anchor=south east]{$o_i$};

%determine a coordinate (O1) using (r,\theta,\phi) coordinates.  This command
%also determines (O1xy), (O1xz), and (O1yz): the xy-, xz-, and yz-projections
%of the point (O1).
%syntax: \tdplotsetcoord{Coordinate name without parentheses}{r}{\theta}{\phi}
\tdplotsetcoord{O1}{\rvec}{\betavec}{\phivec};
\node[label={right:$o_{i+1}$}] at (O1) {};

\tdplotsetcoord{P}{.8}{90}{60};
\node[label={above:$p_i$}] at (P) {};
%draw figure contents
%--------------------

%draw the main coordinate system axes
\draw[thick,->] (0,0,0) -- (1,0,0) node[anchor=north east]{$a_i$};
\draw[thick,->] (0,0,0) -- (0,1,0) node[anchor=north west]{$b_i$};
\draw[thick,->] (0,0,0) -- (0,0,1) node[anchor=south]{$c_i(c_i')$};

%draw a vector from origin to point (O1) 
% \draw[-stealth,color=red] (O) -- (O1);

%draw projection on xy plane, and a connecting line
%\draw[dashed] (O) -- (O1xy);
\draw[dashed] (O1) -- (O1xy);

%draw the angle \phi, and label it
%syntax: \tdplotdrawarc[coordinate frame, draw options]{center point}{r}{angle}{label options}{label}
\tdplotdrawarc{(O)}{0.2}{0}{\phivec}{anchor=north}{$\phi_i$}


%set the rotated coordinate system so the x'-y' plane lies within the
%"theta plane" of the main coordinate system
%syntax: \tdplotsetthetaplanecoords{\phi}
\tdplotsetthetaplanecoords{\phivec}

%draw theta arc and label, using rotated coordinate system
% \tdplotdrawarc[tdplot_rotated_coords]{(0,0,0)}{0.5}{0}{\thetavec}{anchor=south west}{$\theta$}
\draw[dashed] (O1) -- (P);
\tdplotdrawarc[tdplot_rotated_coords]{(P)}{.8}{-90}{-60}{anchor=south west}{$ds$}
\tdplotdrawarc[tdplot_rotated_coords]{(P)}{.2}{-90}{-60}{anchor=south east}{$\theta_i$}

%draw some dashed arcs, demonstrating direct arc drawing
%\draw[dashed,tdplot_rotated_coords] (\rvec,0,0) arc (0:90:\rvec);
%\draw[dashed] (\rvec,0,0) arc (0:90:\rvec);

%set the rotated coordinate definition within display using a translation
%coordinate and Euler angles in the "z(\phi)y(\beta)z(\gamma)" euler rotation convention
%syntax: \tdplotsetrotatedcoords{\phi}{\beta}{\gamma}
%\tdplotsetrotatedcoords{\phivec}{\thetavec}{0}
\tdplotsetrotatedcoords{\phivec}{0}{0}

%translate the rotated coordinate system
%syntax: \tdplotsetrotatedcoordsorigin{point}
%\tdplotsetrotatedcoordsorigin{(O1)}
\tdplotsetrotatedcoordsorigin{(O)}

\draw[thick,tdplot_rotated_coords,->] (0,0,0) -- (.3,0,0) node[anchor=north]{$k_i$};
\draw[dashed,tdplot_rotated_coords,->] (0,0,0) -- (1,0,0) node[anchor=north west]{$a_i'$};
\draw[dashed,tdplot_rotated_coords,->] (0,0,0) -- (0,1,0) node[anchor=west]{$b_i'$};
\draw[dashed] (1,0,0) arc (0:60:1);
\draw[dashed] (0,1,0) arc (90:150:1);

%WARNING:  coordinates defined by the \coordinate command (eg. (O), (O1), etc.)
%cannot be used in rotated coordinate frames.  Use only literal coordinates.  

\tdplotsetrotatedcoords{\phivec}{\thetavec}{0}
\tdplotsetrotatedcoordsorigin{(O1)}

% use the tdplot_rotated_coords style to work in the rotated, translated coordinate frame
\draw[thick,tdplot_rotated_coords,->] (0,0,0) -- (.3,0,0) node[anchor=west]{$a_{i+1}$};
\draw[thick,tdplot_rotated_coords,->] (0,0,0) -- (0,.3,0) node[anchor=west]{$b_{i+1}$};
\draw[thick,tdplot_rotated_coords,->] (0,0,0) -- (0,0,.3) node[anchor=south]{$c_{i+1}$};

%draw some vector, and its projection, in the rotated coordinate frame
%\draw[-stealth,color=blue,tdplot_rotated_coords] (0,0,0) -- (.2,.2,.2);
%\draw[dashed,color=blue,tdplot_rotated_coords] (0,0,0) -- (.2,.2,0);
%\draw[dashed,color=blue,tdplot_rotated_coords] (.2,.2,0) -- (.2,.2,.2);

%show its phi arc and label
%\tdplotdrawarc[tdplot_rotated_coords,color=blue]{(0,0,0)}{0.2}{0}{45}{anchor=north west,color=black}{$\phi'$}

%change the rotated coordinate frame so that it lies in its theta plane.
%Note that this overwrites the original rotated coordinate frame
%syntax: \tdplotsetrotatedthetaplanecoords{\phi'}
%\tdplotsetrotatedthetaplanecoords{45}

%draw theta arc and label
%\tdplotdrawarc[tdplot_rotated_coords,color=blue]{(0,0,0)}{0.2}{0}{55}{anchor=south west,color=black}{$\theta'$}

\end{tikzpicture}

\end{document}