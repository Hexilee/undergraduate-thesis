\documentclass[tikz]{standalone}

% \documentclass{article}

% \usetikzlibrary{calc,3d,arrows}% tikz package already loaded by 'tikz' option
\usepackage{tikz-3dplot}
\usepackage{tikz}   %TikZ is required for this to work.  Make sure this exists before the next line

\usepackage[active,tightpage]{preview}  %generates a tightly fitting border around the work

\PreviewEnvironment{tikzpicture}
\setlength\PreviewBorder{2mm}

\begin{document}

%Angle Definitions
%-----------------
\tdplotsetmaincoords{90}{0}

%define polar coordinates for some vector
%TODO: look into using 3d spherical coordinate system
\pgfmathsetmacro{\thetavec}{30}
\pgfmathsetmacro{\betavec}{15}


%start tikz picture, and use the tdplot_main_coords style to implement the display 
%coordinate transformation provided by 3dplot
\begin{tikzpicture}[scale=7,tdplot_main_coords]

%set up some coordinates 
%-----------------------
\coordinate (O) at (0,0,0) node[anchor=south east]{$o_i$};

\tdplotsetcoord{P}{.8}{90}{0};
\node[label={below:$p_i$}] at (P) {};

\tdplotsetcoord{O1}{.41}{\betavec}{0};
\node[anchor=west] at (O1) {$o_{i+1}$};

%determine a coordinate (O1) using (r,\theta,\phi) coordinates.  This command
%also determines (O1xy), (O1xz), and (O1yz): the xy-, xz-, and yz-projections
%of the point (O1).
%syntax: \tdplotsetcoord{Coordinate name without parentheses}{r}{\theta}{\phi}
%draw figure contents
%--------------------

%draw the main coordinate system axes
\draw[thick,->] (0,0,0) -- (1,0,0) node[label={right:$a_i$}]{};
\draw[thick,->] (0,0,0) -- (0,0,1) node[label={above:$b_i$}]{};


%draw theta arc and label, using rotated coordinate system
% \tdplotdrawarc[tdplot_rotated_coords]{(0,0,0)}{0.5}{0}{\thetavec}{anchor=south west}{$\theta$}
\draw[dashed] (O1) -- (P);

\tdplotsetthetaplanecoords{0}

\tdplotdrawarc[tdplot_rotated_coords]{(P)}{.8}{-90}{-60}{anchor=west}{$ds$}
\tdplotdrawarc[tdplot_rotated_coords]{(P)}{.2}{-90}{-60}{anchor=east}{$\theta_i$}

%draw some dashed arcs, demonstrating direct arc drawing
%\draw[dashed,tdplot_rotated_coords] (\rvec,0,0) arc (0:90:\rvec);
%\draw[dashed] (\rvec,0,0) arc (0:90:\rvec);

%set the rotated coordinate definition within display using a translation
%coordinate and Euler angles in the "z(\phi)y(\beta)z(\gamma)" euler rotation convention
%syntax: \tdplotsetrotatedcoords{\phi}{\beta}{\gamma}
%\tdplotsetrotatedcoords{\phivec}{\thetavec}{0}

\tdplotsetrotatedcoords{0}{\thetavec}{0}
\tdplotsetrotatedcoordsorigin{(O1)}
\begin{scope}[tdplot_rotated_coords]
    % use the tdplot_rotated_coords style to work in the rotated, translated coordinate frame
    \draw[dashed,->] (0,0,0) -- (.3,0,0) node[anchor=west]{$a_{i+1}$};
    \draw[dashed,->] (0,0,0) -- (0,0,.3) node[label={right:$b_{i+1}$}]{};

    \tdplotsetcoord{P1}{.8}{-90}{0};
    \node[label={below:$p_{i+1}$}] at (P1) {};

    \tdplotsetcoord{O2}{.41}{-15}{0};
    \node[anchor=west] at (O2) {$o_{i+2}$};

    \draw[dashed] (O1) -- (P1);
    \draw[dashed] (O2) -- (P1);

    \tdplotsetthetaplanecoords{0}
    \tdplotdrawarc[tdplot_rotated_coords]{(P1)}{.8}{90}{120}{anchor=east}{$ds$}
    \tdplotdrawarc[tdplot_rotated_coords]{(P1)}{.2}{90}{120}{anchor=west}{$\theta_{i+1}$}
\end{scope}

%draw some vector, and its projection, in the rotated coordinate frame
%\draw[-stealth,color=blue,tdplot_rotated_coords] (0,0,0) -- (.2,.2,.2);
%\draw[dashed,color=blue,tdplot_rotated_coords] (0,0,0) -- (.2,.2,0);
%\draw[dashed,color=blue,tdplot_rotated_coords] (.2,.2,0) -- (.2,.2,.2);

%show its phi arc and label
%\tdplotdrawarc[tdplot_rotated_coords,color=blue]{(0,0,0)}{0.2}{0}{45}{anchor=north west,color=black}{$\phi'$}

%change the rotated coordinate frame so that it lies in its theta plane.
%Note that this overwrites the original rotated coordinate frame
%syntax: \tdplotsetrotatedthetaplanecoords{\phi'}
%\tdplotsetrotatedthetaplanecoords{45}

%draw theta arc and label
%\tdplotdrawarc[tdplot_rotated_coords,color=blue]{(0,0,0)}{0.2}{0}{55}{anchor=south west,color=black}{$\theta'$}

\end{tikzpicture}

\end{document}