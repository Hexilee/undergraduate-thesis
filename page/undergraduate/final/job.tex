\clearpage
\sectionnonum{本科生毕业论文(设计)任务书}

{
    \bfseries
    \noindent 一、题目:形变传感器三维可视化软件开发\\
    \noindent 二、指导教师对毕业论文(设计)的进度安排及任务要求:\\

    \begin{enumerate}
        \item 任务要求
        \begin{itemize}
            \item 将光纤光栅(Fiber Bragg Grating, FBG)应变传感阵列测得的线缆形状实时数据处理为数据接口的输入数据。
            \item 编写程序,根据输入数据渲染出线缆的实时形状,并支持基本的用户交互(视角拉近、拖拽等)。
        \end{itemize}
        \item 进度安排
        \begin{itemize}
            \item 11月8日至11月22日,约两周时间,完成 FBG 传感阵列相关的文献阅读,对 FBG 传感阵列直接可测得的数据类型和经过常见处理后可获得的数据类型形成基本的认识。
            \item 11月23日至12月1日,完成3D重建相关内容的文献阅读,了解3D重建所必需的数据类型。
            \item 12月2日至12月15日,约两周时间,确定数据接口并测试其可行性(尝试把 FBG 测得的原始数据转化为接口数据)。
            \item 12月16日至1月1日,约两周时间,小结并准备开题报告。
            \item 1月2日至2月20日,寒假期间,初步实现渲染程序。
            \item 2月21日至2月28日,完成渲染功能并使用理想建模数据进行测试。
            \item 2月29日至3月15日,约两周时间,使用现有柔性曲率传感器的测试数据进行测试并进行误差分析。
            \item 3月16日至3月23日,完成中期答辩。
            \item 3月24日至4月1日,探寻适合获取光纤光栅应变传感阵列所测数据的传输方式及目标平台(如一台 Linux 机器)。
            \item 4月1日至4月15日,约两周时间,在目标平台上处理数据并提供网络接口给客户端。
            \item 4月15日至5月1日,约两周时间,完成剩下的交互功能并构建全平台客户端。
            \item 5月2日 - ,撰写结题报告并准备答辩。

        \end{itemize}
    \end{enumerate}

    起讫日期 2019年11月8日至2020年7月1日
    \begin{flushright}
        \bfseries \zihao{-4}
            指导教师(签名) \underline{\multido{}{5}{\quad}} 职称 \underline{\multido{}{5}{\quad}}
    \end{flushright}

    \noindent 三、系或研究所审核意见:
    
    同意

    \mbox{} \vfill
    \signature{负责人(签名)}
}
