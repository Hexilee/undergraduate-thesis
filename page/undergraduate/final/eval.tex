\clearpage
\sectionnonum{本科生毕业论文(设计)考核}

{
    \bfseries
    \noindent 一、指导教师对毕业论文(设计)的评语:\\

该学生围绕三维图形渲染、三维曲线重建算法和网络数据传输等方面展开了研究,完成了相关的理论论述与软件系统设计,并进行了实验测试。论文成果具有一定的理论意义和实用价值。

论文表明该学生阅读了大量相关的研究文献,已掌握了所研究领域的基本理论与知识。研究问题方法正确,论文中的公式正确,实验测试结果可靠,实验结果对基于曲率数据三维曲线重建的研究方向具有一定促进作用。

论文反映了该学生在光学、数学、计算机等方面具有一定的理论基础和专业知识,具备分析问题和解决问题的能力,达到了本科生的毕业要求。

    \signature{指导教师(签名)}

    \noindent 二、答辩小组对毕业论文(设计)的答辩评语及总评成绩:\\

    本论文符合浙江大学本科生毕设相关要求,予以通过并建议授予学士学位。

    \mbox{} \vfill

    \finaleval[8.1][11.79][3.99][51.33][75]

    \signature{负责人(签名)}
}
