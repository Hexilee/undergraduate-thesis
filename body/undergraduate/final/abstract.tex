\cleardoublepage{}
\begin{center}
    \bfseries \zihao{3} 摘要
\end{center}

光纤传感的出现让实时确定物体的形态成为了一个很有前景的研究方向。
且随着相关行业的发展,医用、工业用光纤形变传感器的落地,形变数据可视化的需求也越来越大。
然而,并没有学者研究如何为形变传感器构建一个通用的三维实时可视化软件。

另一方面,随着互联网革命的持续推进,基于浏览器的图像渲染、数据传输技术逐渐成熟并成为大势所趋。
一款运行在浏览器上的形变传感器可视化软件是时代浪潮下的最佳选择。

本文介绍了一份基于Cartesian运动坐标系重建,
使用WebSocket和HTTPS传输数据,
使用WebGL渲染的形变传感器三维可视化软件实现。

\cleardoublepage{}
\begin{center}
    \bfseries \zihao{3} Abstract
\end{center}

The advent of optical fiber sensing has made real-time determination of the shape of an object a promising research direction. 
With the development of related industries and the deployment of optical fiber deformation sensors for medical and industrial use, 
the demand for visualization of deformation data is also increasing. 
However, no scholars have studied how to build a universal 3D real-time visualization software for the deformation sensor.

On the other hand, with the continuous advancement of the Internet revolution, 
browser-based image rendering and data transmission technologies have gradually matured and become the general trend. 
A deformation sensor visualization software running on a browser is the best choice under the tide of the times.

This article introduces a 3D visualization software implementation of deformation sensor 
based on Cartesian motion coordinate system reconstruction, using WebSocket and HTTPS data push, and WebGL rendering.