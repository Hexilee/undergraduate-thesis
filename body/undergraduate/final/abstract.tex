\clearpage
\begin{center}
    \bfseries \zihao{3} 摘要
\end{center}

光纤传感的出现让实时确定物体的形态成为了一个很有前景的研究方向。
且随着相关行业的发展,医用、工业用光纤形变传感器的落地,形变数据可视化的需求也逐渐显现。
然而,未闻有学者研究如何为形变传感器构建一个通用的形变传感可视化软件。

因此,本文介绍了一种基于传感曲率数据的三维曲线可视化软件构建方案。
其主要模块包括原始数据上推、数据连续化、坐标重建、重建数据下推及渲染;
使用的主要技术包括三维曲线坐标重建、网络数据传输和三维渲染。

为使软件更通用、安全、准确、实时,
本文提出了一种基于Cartesian坐标系和Frenet标架上两种重建算法的改良算法,
研究了基于HTTPS和WebSocket的数据传输技术,
利用WebGL渲染引擎和Threejs库实现了对轴线和管壁的渲染,
并在最后展示了软件的实际运行效果。

\textbf{关键词}:光纤传感;曲线重建;网络传输;三维渲染。

\clearpage

\begin{center}
    \bfseries \zihao{3} Abstract
\end{center}

The advent of optical fiber sensing makes real-time determination of the shape of an object a promising research direction.
And with the development of related industries and the landing of optical fiber deformation sensors for medical and industrial use, the demand for visualization of deformation data has gradually emerged.
However, no scholars have studied how to build a universal deformation sensor visualization software for deformation sensors.

Therefore, this article introduces a three-dimensional curve visualization software construction scheme based on sensor curvature data.
Its main modules include original data push up, data continuity, coordinate reconstruction, reconstructed data push down and rendering;
The main technologies used include 3D curvilinear coordinate reconstruction, network data transmission and 3D rendering.

To make the software more versatile, safe, accurate and real-time,
This paper proposes an improved algorithm based on two reconstruction algorithms on the Cartesian coordinate system and the Frenet frame,
researches the data transmission technology based on HTTPS and WebSocket,
uses WebGL rendering engine and Threejs library to achieve the rendering of the axis and pipe wall,
and at the end it shows the actual running effect of the software.

\textbf{Keywords}: optical fiber sensing; curve reconstruction; network transmission; three-dimensional rendering.