\cleardoublepage

\section{绪论}

形变传感器主要分为两大类,一类是传统形变传感器(Conventional Shape Sensors, CSS),
另一类是光形变传感器(Fiber Optic Shape Sensors, FOSS)。
通常,FOSS与CSS相比具有许多明显而实用的优点,例如:

\begin{itemize}
\item FOSS可以仅由一个远程询问器单元进行监控,而无需布线和连接许多传感器。
\item 传感器的位置不需要电力,因此可以将其放置在传统手段无法接近的地方,并测量这些部位的应变或弯曲。
\item 小尺寸的光纤(直径在100μm和2 mm之间)可以被嵌入非常薄的材料、表面、结构中,或者嵌入到杆或小型设备的中心。
\item 光纤传感器不受外部电磁场的影响。
\end{itemize}

光纤传感的出现让实时确定物体的形态成为了一个很有前景的研究方向\cite{recent-dev-in-foss}。
FOSS利用光纤传感器(Fiber Optic Sensors, FOS)来实现相对位置测量,
或是使用嵌入式FOS实现物体形状的测量。
FOSS被设计用于定向应变测量,例如,
FOSS可以由三芯光纤布拉格光栅(Fiber Bragg Grating, FBG)传感器平面组成,
该传感器平面测量应变以进行对象的多维曲率计算,可以在计算机模型中用于重建对象的2D和3D形状。

随着相关行业的发展,医用、工业用光纤形变传感器的落地,形变数据可视化的需求也越来越大。
然而,并没有学者研究如何为形变传感器构建一个通用的三维实时可视化软件。

另一方面,随着互联网革命的持续推进,基于浏览器的图像渲染、数据传输技术逐渐成熟并成为大势所趋。
一款运行在浏览器上的形变传感器可视化软件是时代浪潮下的最佳选择。

构建运行在浏览器上的可视化软件需要考虑以下几个问题:

\begin{itemize}
\item 如何通过传感器获取的数据重建出曲线的空间各点坐标?
\item 如何根据各点坐标渲染出3D图形?
\item 又如何将服务端重建的点坐标安全又实时地传输到客户端渲染程序?
\end{itemize}

本文将基于各类研究与文献来讨论这些问题,并介绍一份具体的软件实现。


