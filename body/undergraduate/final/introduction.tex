\cleardoublepage

\section{绪论}

形变传感器主要分为两大类,一类是传统形变传感器(Conventional Shape Sensors, CSS),
另一类是光形变传感器(Fiber Optic Shape Sensors, FOSS)。
通常,FOSS与CSS相比具有许多明显而实用的优点,例如:

\begin{itemize}
\item FOSS可以仅由一个远程询问器单元进行监控,而无需布线和连接许多传感器。
\item 传感器的位置不需要电力,因此可以将其放置在传统手段无法接近的地方,并测量这些部位的应变或弯曲。
\item 小尺寸的光纤(直径在100μm和2 mm之间)可以被嵌入非常薄的材料、表面、结构中,或者嵌入到杆或小型设备的中心。
\item 光纤传感器不受外部电磁场的影响。
\end{itemize}

光纤传感的出现让实时确定物体的形态成为了一个很有前景的研究方向\cite{recent-dev-in-foss}。
FOSS利用光纤传感器(Fiber Optic Sensors, FOS)来实现相对位置测量,
或是使用嵌入式FOS实现物体形状的测量。
FOSS被设计用于定向应变测量,例如,
FOSS可以由三芯光纤布拉格光栅(Fiber Bragg Grating, FBG)传感器平面组成,
该传感器平面测量应变以进行对象的多维曲率计算,可以在计算机模型中用于重建对象的2D和3D形状。

随着相关行业的发展,医用、工业用光纤形变传感器的落地,形变数据可视化的需求也越来越大。
然而,并没有学者研究如何为形变传感器构建一个通用的三维实时可视化软件。

另一方面,随着互联网革命的持续推进,基于浏览器的图像渲染、数据传输技术逐渐成熟并成为大势所趋。
一款运行在浏览器上的形变传感器可视化软件是时代浪潮下的最佳选择。

构建运行在浏览器上的可视化软件需要考虑以下几个问题:

\begin{itemize}
\item 如何通过传感器获取的数据重建出曲线的空间各点坐标?
\item 如何根据各点坐标渲染出3D图形?
\item 又如何将服务端重建的点坐标安全又实时地传输到客户端渲染程序?
\end{itemize}

本章将基于各类研究与文献来讨论这些问题,后面几章将介绍一份具体的软件实现。

\subsection{曲线重建}

曲线重建主要分为两个部分,一个是连续化,即将离散的数据连续化;
另一部分是重建,根据连续的数据得到绝对坐标系下的曲线各点坐标。
根据传感器的种类不同,所测得的数据类型不同,这两步工作的执行顺序也可以不同。

如MEMS惯性传感器主要测量加速度数据,并对加速度数据进行二次积分而得到测量点位移,
故可以在计算测量点坐标之后再进行连续化;而像FBG传感器\cite{FBG-sensor-devices}所得到的是测量点的曲率数据,
难以直接计算获得该点坐标,必须先对曲率数据进行连续化,然后再计算得出各点的坐标。

本文主采用更高精度、基于曲率数据的曲线重建方案。故可选用FBG传感器并假定可测得数据为两个正交方向的曲率。

\subsubsection{连续化}
图形学上常用的连续化算法(曲线)有贝塞尔(Bezier)曲线、B样条曲线、Catmull Rom样条曲线等;
而统计学上常用线性插值、多项式插值等。图形学连续化的常见目的是得到光滑曲线,而统计学则追求最小偏差。
由于“曲率光滑程度对曲线重建结果的影响”尚无理论研究,也超出了本文的研究范围,故本文采用其中最简便的线性插值法。

\subsubsection{重建}
常用的重建算法有基于Cartesian坐标系的拟合算法和基于Frenet坐标系的拟合算法。
其中Cartesian坐标系下的拟合算法拥有更高精度,而Frenet坐标系下的拟合算法更易于编程计算。
本文采用Cartesian坐标系。


\subsection{数据传输}
数据传输也主要分为两个部分:一是传感客户端到服务端的原始数据上推流;二是服务器到渲染客户端的重建数据下推流。
本文的数据传输采用 WebSocket on HTTPS,全程加密,流式传输,安全高效。

WebSocket\cite{rfc6455}(简称WS)是浏览器提供的一种与服务器进行全双工通讯的网络技术,
它是一个独立的协议,但支持使用HTTP/1.x\cite{rfc7230}握手。

相比于HTTP/1.x,它有如下优点:

\begin{itemize}
    \item 支持双向通信,实时性更强;
    \item 更好的二进制支持;
    \item 较少的控制开销。连接创建后,WS客户端、服务端进行数据交换时,协议控制的数据包头部较小。在不包含头部的情况下,服务端到客户端的包头只有2~10字节(取决于数据包长度),客户端到服务端的的话,需要加上额外的4字节的掩码。而HTTP/1.x每次通信都需要携带完整的头部;
    \item 支持扩展。WS协议定义了扩展,用户可以扩展协议,或者实现自定义的子协议(比如支持自定义压缩算法等)。
\end{itemize}

超文本传输协议\cite{rfc7230}(Hyper-Text Transfer Protocol, HTTP)默认基于传输控制协议\cite{rfc793}(Transmission Control Protocol, TCP),
而TCP是明文传输且无验证机制的,其存在三大风险:

\begin{enumerate}
    \item 窃听风险(eavesdropping):第三方可以获知通信内容;
    \item 篡改风险(tampering):第三方可以修改通信内容;
    \item 冒充风险(pretending):第三方可以冒充他人身份参与通信(中间人攻击)。
\end{enumerate}

传输层安全协议\cite{rfc8446}(Transport Layer Security, TLS)是为了解决这三大风险而设计的,其设计目标是:

\begin{enumerate}
    \item 所有信息都是加密传播,第三方无法窃听;
    \item 具有校验机制,一旦被篡改,通信双方会立刻发现;
    \item 配备身份证书,防止身份被冒充。
\end{enumerate}

TLS是介于TCP协议和上层应用协议之间的安全层,其作用机理是基于TCP连接建立安全的TLS连接,
再基于TLS连接实现上层协议。基于TLS的HTTP被称为HTTPS。

\subsection{三维渲染}

三维渲染采用WebGL渲染引擎和Threejs库,主要功能是根据服务器下推的点列实时渲染管壁或轴线。

WebGL是浏览器上的OpenGL实现,用于在任何兼容的Web浏览器中呈现交互式3D和2D图形\cite{webgl}。
WebGL通过引入一个与OpenGL ES 2.0紧密相符合的API,可以在HTML5 <canvas>元素中使用。

目前支持WebGL的浏览器有:Firefox 4+、Google Chrome 9+、Opera 12+、Safari 5.1+和Internet Explorer 11+;
但是WebGL一些特性也需要用户的硬件设备支持。

WebGL几乎是完全地“跨平台”,使用任何支持WebGL的浏览器打开网页即可完成客户端渲染,而无需下载另外的软件或浏览器插件。

Three.js是一个基于WebGL的3D图形库\cite{threejs},它封装了很多实用的图形组件。
