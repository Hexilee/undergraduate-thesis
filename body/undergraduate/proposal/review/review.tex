\cleardoublepage
\chapter{文献综述}
\section{引言} 
形变传感器的研究进展和趋势如何?
各类形变传感器各有何优缺点?
在各种应用场景下应使用哪一类形变传感器?
使用某类传感器时可以获得哪一类的数据?
如何通过不同类别的数据重建出曲线的空间各点坐标?
如何根据各点坐标渲染出3D图形?
又如何将服务端重建的点坐标安全又实时地传输到客户端渲染程序?

本章将基于各类研究与文献来讨论这些问题。

\section{国内外研究现状}
\subsection{形变传感器的研究方向及进展}

形变传感器主要分为两大类,一类是传统形变传感器(Conventional Shape Sensors, CSS),
另一类是光纤形变传感器(Fiber Optic Shape Sensors, FOSS), 
其中传统形变传感器又分为非接触式和接触式两种。

非接触式传感器包括视觉系统传感器(如相机)、
无线电监测与测距(Radio Detection and Ranging, RaDAR)
或光监测与测距(Light Detection and Ranging, LiDAR)传感器,
这些传感器的性能与正确性受环境温度和污染干扰很大。

随着传感器的需求场景越来越多且复杂,非接触传感系统越来越局限,
对小型且灵活的接触式传感器的需求也就越来越大。
接触式传感器可以直接连接到物体上随其移动,
并将位置转换为光、电信号以感测形状、曲率、弯曲和扭曲。

\subsubsection{接触式传统形变传感器}

接触式CSS主要分为电阻压力传感器、光电传感器和微机电系统(Micro-Electo-Mechanical System, MEMS)传感器。
它们各自的简介、优缺点和应用如下表\cite{recent-dev-in-foss}:

\begin{table}[!htbp]\small
\begin{center}
\begin{tabular}{p{0.20\textwidth}p{0.20\textwidth}p{0.30\textwidth}p{0.30\textwidth}}
\toprule
\textbf{传感器技术} & \textbf{简介} & \makebox[5cm][c]{\textbf{优点与应用}} & \makebox[5cm][c]{\textbf{缺点}}\\

\midrule

电阻压力传感器 & 用于短距离二维弯曲或关节运动测量。&
\begin{itemize}
\setlength{\itemsep}{0pt}
\setlength{\parsep}{0pt}
\setlength{\parskip}{0pt}
    \item 低成本;
    \item 与其它电器件相兼容;
    \item 非常适用于可穿戴电子产品。
\end{itemize}
& 
\begin{itemize}
\setlength{\itemsep}{0pt}
\setlength{\parsep}{0pt}
\setlength{\parskip}{0pt}
    \item 不够精准;
    \item 尺寸重量过大,不适用于小尺寸的自动控制应用;
    \item 复杂且接线繁琐,不适用于大规模应用。
\end{itemize} \\

\midrule

光电传感器 & 基于密集的传感器网络工作:一种可计算物体3D表面的算法解决方案;一种电阻率传感器网络;三轴陀螺仪、三轴加速 度计和航班时间距离传感器。&
\begin{itemize}
\setlength{\itemsep}{0pt}
\setlength{\parsep}{0pt}
\setlength{\parskip}{0pt}
    \item 作为可穿戴设备检测脊椎姿势变化,进行医疗运动和预防伤害方面具有巨大潜力;
    \item 易于大规模生产;
    \item 与Arduinos等价格低廉的微电子设备兼容。
\end{itemize}
& 
\begin{itemize}
\setlength{\itemsep}{0pt}
\setlength{\parsep}{0pt}
\setlength{\parskip}{0pt}
    \item 对于工业应用而言,不是可靠且准确的解决方案;
    \item 要求传感器周围有自由空间,以便光学传感器能够正常工作;
    \item 无法保证精度;
    \item 不够灵活。
\end{itemize} \\

\midrule

MEMS传感器 & 结合了加速度计、陀螺仪和磁力计的微电机系统。&
\begin{itemize}
\setlength{\itemsep}{0pt}
\setlength{\parsep}{0pt}
\setlength{\parskip}{0pt}
    \item 钻井;
    \item 地面监控;
    \item 远程测量;
    \item 适用于密闭空间;
    \item 精加工。
\end{itemize}
& 
\begin{itemize}
\setlength{\itemsep}{0pt}
\setlength{\parsep}{0pt}
\setlength{\parskip}{0pt}
    \item 难以定制;
    \item 笨重;
    \item 低柔韧性(长关节);
    \item 测量分辨率低;;
    \item 昂贵;
\end{itemize} \\
\bottomrule
\end{tabular}
\caption{接触式CSS的优缺点对比}
\end{center}
\end{table}

分析可知这三种 CSS 传感器各自的适用场景:

\begin{itemize}
    \item 电阻压力传感器适用于消费级电子产品,如游戏用压感器、心率检测仪等。
    \item 光电传感器适用于通用基础设备,如医疗运动用人体肌肉脊椎形状传感。
    \item MEMS传感器适用于大型工程,如钻井、海床监控等。
\end{itemize}

\subsubsection{光纤形变传感器}

光纤传感的出现让实时确定物体的形态成为了一个很有前景的研究方向。
FOSS利用光纤传感器(Fiber Optic Sensors, FOS)来实现相对位置测量,
或是使用嵌入式FOS实现物体形状的测量。
FOSS被设计用于定向应变测量,例如,
FOSS可以由三芯光纤布拉格光栅(Fiber Bragg Grating, FBG)传感器平面组成,
该传感器平面测量应变以进行对象的多维曲率计算,因此可以在计算机模型中用于重建对象的2D和3D形状。

通常,FOSS与CSS相比具有许多明显而实用的优点,例如:

\begin{itemize}
\item FOSS可以仅由一个远程询问器单元进行监控,而无需布线和连接许多传感器。
\item 传感器的位置不需要电力,因此可以将其放置在传统手段无法接近的地方,并测量这些部位的应变或弯曲。
\item 小尺寸的光纤(直径在100μm和2 mm之间)可以被嵌入非常薄的材料、表面、结构中,或者嵌入到杆或小型设备的中心。
\item 光纤传感器不受外部电磁场的影响。
\end{itemize}

FOS的发展趋势如下表\cite{recent-dev-in-foss}:

\begin{table}[!htbp]\small
\begin{center}
\begin{tabular}{p{0.20\textwidth}cp{0.80\textwidth}}
\toprule
\textbf{进展} & \textbf{年份} & \makebox[5cm][c]{\textbf{简介}}\\

\midrule

弯曲传感器 & \textasciitilde 1980 & 确定了三种主要的早期弯曲感测方法。它们分别是:
通过正常光纤传输的功率变化感测;
通过配备了弯曲调制器的光纤传输功率变化来感测;
通过同一包层中并行核心之间的串扰而导致的传输功率变化来感测。
\\
\\
定向弯曲传感器 & \textasciitilde 1990 & 通过使用多根单独的光纤和多芯光纤,可以检测光纤弯曲方向。
其采用了多种技术,如单个的FBG、宏弯曲调制器、多芯光子晶体光纤模场变化和弯曲引起的芯串扰。
\\
\\
分布式弯曲传感 & \textasciitilde 1996 & 通过多路复用,光纤能够通过FBG在多个位置检测应变。 
最初,只有波分和时分多路复用用于光纤形状感测,这限制了传感器的数量和最小的弯曲半径。
\\
\\
第一款商用3D光纤形变传感(ShapeTape)& \textasciitilde 1999 & ShapeTape是最早用于计算物体形状和表面的光纤3D测量设备之一。 
它是在1990年代基于光纤压力传感器开发的。 
由于价格高,范围、灵活性和准确性都很有限,ShapeTape在商业上并不成功。
\\
\\
基础光纤形变传感 & \textasciitilde 2000 & 通过多路复用检测多个位置处的弯曲,为三维形状感测提供了机会。
三个或更多芯的光纤提供了在多个位置检测方向应变的能力,从而可以重构空间中的光缆形状。
\\
\\
分布式光线形变传感 & \textasciitilde 2004 & 在多个磁芯上数千个应变位置的检测使连续光栅感测三维形状成为可能,
从而消除了先前形状感测技术的离散性质。 
光频域反射计(Optical Frequency Domain Reflectometry, OFDR)可以利用普通光纤和连续光栅光纤中的瑞利反向散射光进行形状感测。
\\
\\
用于工业和医疗机器人的光纤形变传感 & \textasciitilde 2006 & 各种专利的出现促使着工业和医疗器械中FOSS的使用,
这似乎是工业界第一次在实际应用中对FOSS表现出兴趣。
\\
\\
螺旋光纤布拉格光栅传感器 & \textasciitilde 2008 & 这篇论文似乎是对多芯光纤中使用螺旋FBG进行扭曲和弯曲测量的第一个演示,
这为如何基于FOSS实现完整的3D重建提供了重大改进思路。
\\
\\
光纤形变传感可视化 & \textasciitilde 2009 & 随着返回信号愈发复杂,渲染出使用光纤形状感测的电缆变得更具挑战性。
自2009年以来,越来越多的论文和专利概述了实时渲染FOSS输出的各种方法。
\\
\\
基于光纤形变传感器的力传感 & \textasciitilde 2013 & 使用光纤形状感测电缆测量力不仅可以检测光纤的位置,
还可以检测该点的力。这些论文和专利自2013年以来由医疗器械公司出版,
为多功能形状传感器开辟了道路。
\\
\\
基于多芯光纤和分布式FOSS的分布式布里渊散射 & \textasciitilde 2016 & 首次演示了通过多芯光纤(Multi-Core Fiber, MCF)
中偏心纤芯的布里渊频移来测量长光纤(1km)中的曲率,为远程形状感测带来了新的可能性。
\\
\\
螺旋多芯光纤中的连续光栅 & \textasciitilde 2017 & 关于连续光栅多芯光纤的研究有了新的进展。
复合多芯光纤的商业可用性对于开发高分辨率和分布式FOSS来说至关重要。
\\
\bottomrule
\end{tabular}
\caption{FOSS的发展趋势}
\end{center}
\end{table}

\subsection{曲线重建算法}
曲线重建主要分为两个部分,一个是连续化,即将离散的数据连续化;
另一部分是拟合,根据数据拟合得到绝对坐标系下的曲线各点坐标。
根据传感器的种类不同,所测得的数据类型不同,这两部工作的执行顺序也可以不同。

如MEMS惯性传感器主要测量加速度数据,并对加速度数据进行二次积分而得到测量点位移,
故可以在计算测量点坐标之后再进行连续化;而像FBG传感器所得到的是测量点的曲率数据,
难以直接计算获得该点坐标,必须先对曲率数据进行连续化,然后通过拟合算法计算得出各点的坐标。

本小结主要讨论连续化算法和基于曲率数据的曲线拟合算法。

\subsubsection{连续化算法}

常用的连续化算法有线性插值法、
贝塞尔(Bezier)曲线插值法、
B样条曲线插值法和三次样条曲线插值法。

\begin{enumerate}[label=(\Alph*)]
    \item \textbf{线性插值法} \\
    线性插值法即认为每两个测量点处的曲率变化是线性的,即:
    \begin{equation}
    \left\{
        \begin{array}{lr}
        k_{i, i+1} (S) = a_{i, i+1} (S - S_i) + k_i, \ S_i\leq S\leq S_{i+1}\\
    \\
        a_{i, i+1} = \frac{k_{i+1} - k_i}{S_{i+1} - S_i}
    \\
        k_0 = S_0 = 0
        \end{array}
    \right.
    \end{equation}

    线性插值算法简单,计算速度快;但是分割点连接处不平滑,跟实际产生的曲线可能存在一定的差距。

    \item \textbf{贝塞尔曲线插值法} \\
    贝塞尔曲线的一般式如下:

    \begin{equation}
        k(S) = \Sigma_{i=0}^nC_n ^ i a_i(1-S)^{n - i}S^i
    \end{equation}

    贝塞尔曲线插值结果很光滑,但存在另外的问题:

    \begin{enumerate}[label=(\alph*)]
        \item 确定了测量点数$n+1$,也就决定了所定义的Bezier曲线的阶次$n$,很不灵活。
        \item 当测量阶次$n$较大时,曲线的阶次将比较高。此时,测量点对曲线形状的控制将明显减弱。
        \item Bezier的调和函数$\Sigma_{i=0}^nC_n ^ i (1-S)^{n - i}S^i$的值在开区间$(0,1)$内均不为0。
        因此,所定义的曲线在$(0<S<1)$的区间内的任何一点均要受到全部顶点的影响,
        即改变其中任一个顶点的位置,都将对整条曲线产生影响,因此对曲线进行局部修改是不可能的。
    \end{enumerate}

    \item \textbf{B样条曲线插值法} \\
    为了克服以上提到的在Bezier曲线中存在的问题,
    Gordon、Riesenfeld和Forrest等人拓展了Bezier曲线,用n次B样条基函数替换了伯恩斯坦基函数,
    构造了B样条曲线。B样条曲线除了保持Bezier曲线所具有的优点外,还增加了可以对曲线进行局部修改这一突出的优点。
    除此之外,它还具有对特征多边形更逼近以及多项式阶次较低等优点。因此,B样条曲线在外形设计中得到了更广泛的重视和应用。

    \begin{equation}
    \left\{
        \begin{array}{lr}
        k(S) = \Sigma_{i=0}^n a_i B_{i, deg}(S)
    \\
        B_{i, deg}(S) = \frac{S-S_i}{S_{i+deg} - S_i}B_{i, deg-1}(S) +\frac{S_{i + deg +1}-S}{S_{i + deg +1} - S_i}B_{i+1, deg-1}(S) 
        \end{array}
    \right.
    \end{equation}

    上面就是B样条曲线的方程,其中$B(S)$被称为基础函数表,
    基础函数表本质上是个递归方程,同时也是一个中间参数。
    当前$deg$阶的元素需要通过两个$deg-1$阶的元素获得,
    $deg-1$阶的元素则需要通过$deg-2$阶的元素获得……以此类推,
    直到递归$deg$次以后,回退到0阶为止。

    但是,回退到0阶的时候怎么算呢?B样条算法规定,回退到0阶时使用以下公式: 

    \begin{equation}
    B_{i, 0} = \left\{
        \begin{array}{lr}
        1, \ S_i \le S \le S_{i+1}
    \\
        0, \ S \textless S_i \ or \ S \textgreater S_{i+1}
        \end{array}
    \right.
    \end{equation}

    B样条曲线克服了Bezier曲线遇到的问题,
    但其在多个离散曲率检测点的曲率数据不全在拟合出的曲率连续化曲线上,
    这会产生一定的拟合误差。我们可以采用三次样条曲线插值法来解决这个问题。

    \item \textbf{三次样条曲线插值法} \\
    假设曲率和弧长的关系为:

    \begin{equation}
        k(S) =  a_i + b_i (S - S_i) + c_i  (S - S_i)^2 + d_i (S - S_i)^3
    \end{equation}

    且曲线在第 i 个分割点满足方程:

    \begin{equation}
    \left\{
        \begin{array}{lr}
        k_i(S_i) = C_i
    \\
        k_i(S_{i+1}) = C_{i+1}
    \\
        k_i'(S_{i+1}) = k_{i+1}'(S_{i+1})
    \\
        k_i''(S_{i+1}) = k_{i+1}''(S_{i+1})
        \end{array}
    \right.
    \end{equation}

    即曲率原函数、一阶和二阶微分均满足连续性关系,这就能保证拟合出的曲率曲线是足够光滑的。

    再将此方程组与下式联立:

    \begin{equation}
    \left\{
        \begin{array}{lr}
        k(0) = 0
    \\
        k(S_n) = 0
        \end{array}
    \right.
    \end{equation}

    则可求得$a_i, b_i, c_i, d_i$四个系数的值,最终可通过曲线上每一个点距原点的弧长计算曲率。
\end{enumerate}

\subsubsection{拟合算法}

常用的拟合算法有曲率方向作为运动坐标系坐标轴的拟合算法和基于Frenet标架的拟合算法。

\begin{enumerate}[label=(\Alph*)]
    \item \textbf{曲率方向作为运动坐标系坐标轴的拟合算法} \\

    \begin{figure}
    \centering
    \includegraphics[scale=0.4]{cartesian-coordinate-system.png}
    \caption{曲率方向作为运动坐标系坐标轴}
    \end{figure}

    此算法的原理是以每个微段的方向为z轴建立运动坐标系$o_i-a_ib_ic_i$,
    并通过几何方法求得下一个点在此坐标系下的相对坐标$o_{i+1} \{d_{ai}, d_{bi}, d_{ci}\}$,
    再通过线性变换得到下一个点的运动坐标系$o_{i+1}-a_{i+1}b_{i+1} c_{i+1}$。
    这样就可以求得每个点在上个运动坐标系下的相对坐标以及当前坐标系相对上个坐标系的变换矩阵$t_{i, i+1}$。

    \begin{align}
        k_i &= \sqrt{k_{ai} ^ 2 + k_{bi} ^ 2}, \\
        \theta_i &= k_i \cdot d_s, \\
        cos\alpha_i &= k_{ai} / k_i, \\
        sin\alpha_i &= k_{bi} / k_i, \\
        d_a &= \frac{cos\alpha_i \cdot (1 - cos\theta_i)}{k_i}, \\
        d_b &= \frac{sin\alpha_i \cdot (1 - cos\theta_i)}{k_i}, \\
        d_c &= \frac{sin\theta}{k_i}, \\
        t_{i, i+1} &= \left[
            \begin{matrix}
                cos \alpha_i & -sin \alpha_i & 0 \\
                sin \alpha_i & cos \alpha_i & 0 \\
                0 & 0 & 1
            \end{matrix}
            \right]
            \cdot
            \left[
            \begin{matrix}
                cos \theta_i & 0 & sin \theta_i \\
                0 & 1 & 0 \\
                -sin \theta_i & 0 & cos \theta_i \\
            \end{matrix}
            \right]
            \cdot
            \left[
            \begin{matrix}
                cos \alpha_i & sin \alpha_i & 0 \\
                -sin \alpha_i & cos \alpha_i & 0 \\
                0 & 0 & 1
            \end{matrix}
            \right]
    \end{align}

    当前坐标系点乘三个矩阵的几何意义分别为:\\
    \begin{enumerate}
        \item 绕$\vec c_i$旋转一个$\alpha_i$角使得$\vec a_i$与$\vec k_i$同向;
        \item 绕$\vec b_i$旋转一个$\theta_i$角得到$\vec c_{i+1}$;
        \item 绕$\vec c_{i+1}$旋转一个$-\alpha_i$角得到$\vec a_{i+1}$和$\vec b_{i+1}$。
    \end{enumerate}

    将变换矩阵累乘即可得到各运动坐标系相对绝对(原点)坐标系的变换矩阵 $T_i$。
    然后将相对坐标$o_{i+1} \{d_{ai}, d_{bi}, d_{ci}\}$乘以$T_i$的逆矩阵即可得绝对坐标。

    \item \textbf{基于Frenet标架的拟合算法} \\
    我们也可以使用微分几何中常用的Frenet坐标标架。令$\vec T$为切向量,$\vec N$为主法向量,指向微段方向;$\vec B$为副法向量,
    则有:$\vec B = \vec T \times \vec N$。

    假设第$i$段曲线在$xoz$平面内,则两端点平移量$p_{i+1}$有:
    \begin{equation}
        p_{i+1} = [-(1-cos \ \theta_i)/k_i \quad 0 \quad sin \ \theta_i/k_i]^T
    \end{equation}
    曲率$k_i$、弧长$d_s$和弧角$\theta_i$满足$\theta_i = d_s \cdot k_i$。坐标系平移后沿$B_i$轴旋转$\theta_i$,则有:
    \begin{equation}
    [T_{B_i} ^ {\theta_i}] = \left[
        \begin{matrix}
        cos \theta_i & 0 & -sin \theta_i & 0 \\
        0 &1 & 0 & 0 \\
        sin \theta_i & 0 & cos \theta_i & 0 \\
        0 & 0 & 0 & 1 \\
        \end{matrix}
    \right]
    \end{equation}
    接着将坐标系沿$T_i$旋转$\Delta \phi_{i+1}$,其值由下式求出:
    \begin{equation}
    \Delta \phi_{i+1} = \begin{cases}
        \phi_{i+1} - \phi_i,&(\phi_{i+1} \ge \phi_i) \\
        2\pi + \phi_{i+1} - \phi_i,&(\phi_{i+1} \textless \phi_i)\\
    \end{cases}
    \end{equation}
    则有:
    \begin{equation}
    [T_{T_i} ^ {\Delta\phi_{i+1}}] = \left[
        \begin{matrix}
        cos \Delta\phi_{i+1} & -sin \Delta\phi_{i+1} & 0 & 0 \\
        sin \Delta\phi_{i+1} & cos \Delta\phi_{i+1} & 0 & 0 \\
        0 & 0 & 1 & 0 \\
        0 & 0 & 0 & 1 \\
        \end{matrix}
    \right]
    \end{equation}
    由此可求得一个微段运动坐标系的变换矩阵:
    \begin{equation}
    [t_{i+1}] = \left[
        \begin{matrix}
        & [T_{i+1}] & & p_{i+1} \\
        0 & 0 & 0 & 1 \\
        \end{matrix}
    \right] \tag{3} 
    \end{equation}
    其中$[T_{i+1}] = [T_{B_i} ^ {\theta_i}]  [T_{T_i} ^ {\Delta\phi_{i+1}}]$。

    通过此变换矩阵即可求得每个点的运动坐标系和下个点的相对坐标,再通过变换矩阵的逆矩阵求得绝对坐标,最终拟合出曲线。

\end{enumerate}

\subsection{三维渲染技术}
\subsubsection{OpenGL}
\subsubsection{WebGL和Three.js}
\subsection{网络数据传输技术}
\subsubsection{HTTP}
\subsubsection{WebSocket}
\subsubsection{gRPC}
\subsubsection{TLS}
\section{研究展望}
\newpage
\printbibliography[title={参考文献}]
