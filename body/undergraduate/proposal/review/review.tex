\cleardoublepage
\chapter{文献综述}
\section{引言} 
形变传感器的研究进展和趋势如何?
各类形变传感器各有何优缺点?
在各种应用场景下应使用哪一类形变传感器?
使用某类传感器时可以获得哪一类的数据?
如何通过不同类别的数据重建出曲线的空间各点坐标?
如何根据各点坐标渲染出3D图形?
又如何将服务端重建的点坐标安全又实时地传输到客户端渲染程序?

本章将基于各类研究与文献来讨论这些问题。

\section{国内外研究现状}
\subsection{形变传感器的研究方向及进展}

形变传感器主要分为两大类,一类是传统形变传感器(Conventional Shape Sensors, CSS),
另一类是光纤形变传感器(Fiber Optic Shape Sensors, FOSS), 
其中传统形变传感器又分为非接触式和接触式两种。

非接触式传感器包括视觉系统传感器(如相机)、
无线电监测与测距(Radio Detection and Ranging, RaDAR)
或光监测与测距(Light Detection and Ranging, LiDAR)传感器,
这些传感器的性能与正确性受环境温度和污染干扰很大。

随着传感器的需求场景越来越多且复杂,非接触传感系统越来越局限,
对小型且灵活的接触式传感器的需求也就越来越大。
接触式传感器可以直接连接到物体上随其移动,
并将位置转换为光、电信号以感测形状、曲率、弯曲和扭曲。

\subsubsection{接触式传统形变传感器}

接触式CSS主要分为电阻压力传感器、光电传感器和微机电系统(Micro-Electo-Mechanical System, MEMS)传感器。
它们各自的简介、优缺点和应用如下表\cite{recent-dev-in-foss}:

\begin{table}[!htbp]
\begin{center}
\begin{tabular}{p{0.20\textwidth}p{0.20\textwidth}p{0.30\textwidth}p{0.30\textwidth}}
\toprule
\textbf{传感器技术} & \textbf{简介} & \makebox[5cm][c]{\textbf{优点与应用}} & \makebox[5cm][c]{\textbf{缺点}}\\

\midrule

电阻压力传感器 & 用于短距离二维弯曲或关节运动测量。&
\begin{itemize}
\setlength{\itemsep}{0pt}
\setlength{\parsep}{0pt}
\setlength{\parskip}{0pt}
    \item 低成本;
    \item 与其它电器件相兼容;
    \item 非常适用于可穿戴电子产品。
\end{itemize}
& 
\begin{itemize}
\setlength{\itemsep}{0pt}
\setlength{\parsep}{0pt}
\setlength{\parskip}{0pt}
    \item 不够精准;
    \item 尺寸重量过大,不适用于小尺寸的自动控制应用;
    \item 复杂且接线繁琐,不适用于大规模应用。
\end{itemize} \\

\midrule

光电传感器 & 基于密集的传感器网络工作:一种可计算物体3D表面的算法解决方案;一种电阻率传感器网络;三轴陀螺仪、三轴加速 度计和航班时间距离传感器。&
\begin{itemize}
\setlength{\itemsep}{0pt}
\setlength{\parsep}{0pt}
\setlength{\parskip}{0pt}
    \item 作为可穿戴设备检测脊椎姿势变化,进行医疗运动和预防伤害方面具有巨大潜力;
    \item 易于大规模生产;
    \item 与Arduinos等价格低廉的微电子设备兼容。
\end{itemize}
& 
\begin{itemize}
\setlength{\itemsep}{0pt}
\setlength{\parsep}{0pt}
\setlength{\parskip}{0pt}
    \item 对于工业应用而言,不是可靠且准确的解决方案;
    \item 要求传感器周围有自由空间,以便光学传感器能够正常工作;
    \item 无法保证精度;
    \item 不够灵活。
\end{itemize} \\

\midrule

MEMS传感器 & 结合了加速度计、陀螺仪和磁力计的微电机系统。&
\begin{itemize}
\setlength{\itemsep}{0pt}
\setlength{\parsep}{0pt}
\setlength{\parskip}{0pt}
    \item 钻井;
    \item 地面监控;
    \item 远程测量;
    \item 适用于密闭空间;
    \item 精加工。
\end{itemize}
& 
\begin{itemize}
\setlength{\itemsep}{0pt}
\setlength{\parsep}{0pt}
\setlength{\parskip}{0pt}
    \item 难以定制;
    \item 笨重;
    \item 低柔韧性(长关节);
    \item 测量分辨率低;;
    \item 昂贵;
\end{itemize} \\
\bottomrule
\end{tabular}
\caption{接触式CSS的优缺点对比}
\end{center}
\end{table}

\subsubsection{光纤形变传感器}
\subsection{曲线重建算法}
\subsubsection{连续化算法}
\subsubsection{拟合算法}
\subsection{三维渲染技术}
\subsubsection{OpenGL}
\subsubsection{WebGL和Three.js}
\subsection{网络数据传输技术}
\subsubsection{HTTP}
\subsubsection{WebSocket}
\subsubsection{gRPC}
\subsubsection{TLS}
\section{研究展望}
\newpage
\printbibliography[title={参考文献}]
