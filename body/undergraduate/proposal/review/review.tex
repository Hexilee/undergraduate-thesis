\cleardoublepage
\chapter{文献综述}
\section{引言} 
形变传感器的研究进展和趋势如何?
各类形变传感器各有何优缺点?
在各种应用场景下应使用哪一类形变传感器?
使用某类传感器时可以获得哪一类的数据?
如何通过不同类别的数据重建出曲线的空间各点坐标?
如何根据各点坐标渲染出3D图形?
又如何将服务端重建的点坐标安全又实时地传输到客户端渲染程序?

本章将基于各类研究与文献来讨论这些问题。

\section{国内外研究现状}
\subsection{形变传感器的研究方向及进展}

形变传感器主要分为两大类,一类是传统形变传感器(Conventional Shape Sensors, CSS),
另一类是光纤形变传感器(Fiber Optic Shape Sensors, FOSS), 
其中传统形变传感器又分为非接触式和接触式两种。

非接触式传感器包括视觉系统传感器(如相机)、
无线电监测与测距(Radio Detection and Ranging, RaDAR)
或光监测与测距(Light Detection and Ranging, LiDAR)传感器,
这些传感器的性能与正确性受环境温度和污染干扰很大。

随着传感器的需求场景越来越多且复杂,非接触传感系统越来越局限,
对小型且灵活的接触式传感器的需求也就越来越大。
接触式传感器可以直接连接到物体上随其移动,
并将位置转换为光、电信号以感测形状、曲率、弯曲和扭曲。

\subsubsection{接触式传统形变传感器}

接触式CSS主要分为电阻压力传感器、光电传感器和微机电系统(Micro-Electo-Mechanical System, MEMS)传感器。
它们各自的简介、优缺点和应用如下表\cite{recent-dev-in-foss}:

\begin{table}[!htbp]
\begin{center}
\begin{tabular}{p{0.20\textwidth}p{0.20\textwidth}p{0.30\textwidth}p{0.30\textwidth}}
\toprule
\textbf{传感器技术} & \textbf{简介} & \makebox[5cm][c]{\textbf{优点与应用}} & \makebox[5cm][c]{\textbf{缺点}}\\

\midrule

电阻压力传感器 & 用于短距离二维弯曲或关节运动测量。&
\begin{itemize}
\setlength{\itemsep}{0pt}
\setlength{\parsep}{0pt}
\setlength{\parskip}{0pt}
    \item 低成本;
    \item 与其它电器件相兼容;
    \item 非常适用于可穿戴电子产品。
\end{itemize}
& 
\begin{itemize}
\setlength{\itemsep}{0pt}
\setlength{\parsep}{0pt}
\setlength{\parskip}{0pt}
    \item 不够精准;
    \item 尺寸重量过大,不适用于小尺寸的自动控制应用;
    \item 复杂且接线繁琐,不适用于大规模应用。
\end{itemize} \\

\midrule

光电传感器 & 基于密集的传感器网络工作:一种可计算物体3D表面的算法解决方案;一种电阻率传感器网络;三轴陀螺仪、三轴加速 度计和航班时间距离传感器。&
\begin{itemize}
\setlength{\itemsep}{0pt}
\setlength{\parsep}{0pt}
\setlength{\parskip}{0pt}
    \item 作为可穿戴设备检测脊椎姿势变化,进行医疗运动和预防伤害方面具有巨大潜力;
    \item 易于大规模生产;
    \item 与Arduinos等价格低廉的微电子设备兼容。
\end{itemize}
& 
\begin{itemize}
\setlength{\itemsep}{0pt}
\setlength{\parsep}{0pt}
\setlength{\parskip}{0pt}
    \item 对于工业应用而言,不是可靠且准确的解决方案;
    \item 要求传感器周围有自由空间,以便光学传感器能够正常工作;
    \item 无法保证精度;
    \item 不够灵活。
\end{itemize} \\

\midrule

MEMS传感器 & 结合了加速度计、陀螺仪和磁力计的微电机系统。&
\begin{itemize}
\setlength{\itemsep}{0pt}
\setlength{\parsep}{0pt}
\setlength{\parskip}{0pt}
    \item 钻井;
    \item 地面监控;
    \item 远程测量;
    \item 适用于密闭空间;
    \item 精加工。
\end{itemize}
& 
\begin{itemize}
\setlength{\itemsep}{0pt}
\setlength{\parsep}{0pt}
\setlength{\parskip}{0pt}
    \item 难以定制;
    \item 笨重;
    \item 低柔韧性(长关节);
    \item 测量分辨率低;;
    \item 昂贵;
\end{itemize} \\
\bottomrule
\end{tabular}
\caption{接触式CSS的优缺点对比}
\end{center}
\end{table}

分析可知这三种 CSS 传感器各自的适用场景:

\begin{itemize}
    \item 电阻压力传感器适用于消费级电子产品,如游戏用压感器、心率检测仪等。
    \item 光电传感器适用于通用基础设备,如医疗运动用人体肌肉脊椎形状传感。
    \item MEMS传感器适用于大型工程,如钻井、海床监控等。
\end{itemize}

\subsubsection{光纤形变传感器}

光纤传感的出现让实时确定物体的形态成为了一个很有前景的研究方向。
FOSS利用光纤传感器(Fiber Optic Sensors, FOS)来实现相对位置测量,
或是使用嵌入式FOS实现物体形状的测量。
FOSS被设计用于定向应变测量,例如,
FOSS可以由三芯光纤布拉格光栅(Fiber Bragg Grating, FBG)传感器平面组成,
该传感器平面测量应变以进行对象的多维曲率计算,因此可以在计算机模型中用于重建对象的2D和3D形状。

通常,FOSS与CSS相比具有许多明显而实用的优点,例如:

\begin{itemize}
\item FOSS可以仅由一个远程询问器单元进行监控,而无需布线和连接许多传感器。
\item 传感器的位置不需要电力,因此可以将其放置在传统手段无法接近的地方,并测量这些部位的应变或弯曲。
\item 小尺寸的光纤(直径在100μm和2 mm之间)可以被嵌入非常薄的材料、表面、结构中,或者嵌入到杆或小型设备的中心。
\item 光纤传感器不受外部电磁场的影响。
\end{itemize}

FOS的发展趋势如下表\cite{recent-dev-in-foss}:

\begin{table}[!htbp]
\begin{center}
\begin{tabular}{p{0.20\textwidth}cp{0.80\textwidth}}
\toprule
\textbf{进展} & \textbf{年份} & \makebox[5cm][c]{\textbf{简介}}\\

\midrule

弯曲传感器 & ~1980 & 确定了三种主要的早期弯曲感测方法。它们分别是:
通过正常光纤传输的功率变化感测;
通过配备了弯曲调制器的光纤传输功率变化来感测;
通过同一包层中并行核心之间的串扰而导致的传输功率变化来感测。
\\
定向弯曲传感器 & ~1990 & 通过使用多根单独的光纤和多芯光纤,可以检测光纤弯曲方向。
其采用了多种技术,如单个的FBG、宏弯曲调制器、多芯光子晶体光纤模场变化和弯曲引起的芯串扰。
\\
分布式弯曲传感 & ~1996 & 通过多路复用,光纤能够通过FBG在多个位置检测应变。 
最初,只有波分和时分多路复用用于光纤形状感测,这限制了传感器的数量和最小的弯曲半径。
\\
第一款商用3D光纤形变传感(ShapeTape)& ~1999 & ShapeTape是最早用于计算物体形状和表面的光纤3D测量设备之一。 
它是在1990年代基于光纤压力传感器开发的。 
由于价格高,范围、灵活性和准确性都很有限,ShapeTape在商业上并不成功。
\\
基础光纤形变传感 & ~2000 & 通过多路复用检测多个位置处的弯曲,为三维形状感测提供了机会。
三个或更多芯的光纤提供了在多个位置检测方向应变的能力,从而可以重构空间中的光缆形状。
\\
分布式光线形变传感 & ~2004 & 在多个磁芯上数千个应变位置的检测使连续光栅感测三维形状成为可能,
从而消除了先前形状感测技术的离散性质。 
光频域反射计(Optical Frequency Domain Reflectometry, OFDR)可以利用普通光纤和连续光栅光纤中的瑞利反向散射光进行形状感测。
\\
用于工业和医疗机器人的光纤形变传感 & ~2006 & 各种专利的出现促使着工业和医疗器械中FOSS的使用,
这似乎是工业界第一次在实际应用中对FOSS表现出兴趣。
\\
螺旋光纤布拉格光栅传感器 & ~2008 & 这篇论文似乎是对多芯光纤中使用螺旋FBG进行扭曲和弯曲测量的第一个演示,
这为如何基于FOSS实现完整的3D重建提供了重大改进思路。
\\
光纤形变传感可视化 & ~2009 & 随着返回信号愈发复杂,渲染出使用光纤形状感测的电缆变得更具挑战性。
自2009年以来,越来越多的论文和专利概述了实时渲染FOSS输出的各种方法。
\\
基于光纤形变传感器的力传感 & ~2013 & 使用光纤形状感测电缆测量力不仅可以检测光纤的位置,
还可以检测该点的力。这些论文和专利自2013年以来由医疗器械公司出版,
为多功能形状传感器开辟了道路。
\\
基于多芯光纤和分布式FOSS的分布式布里渊散射 & ~2016 & 首次演示了通过多芯光纤(Multi-Core Fiber, MCF)
中偏心纤芯的布里渊频移来测量长光纤(1km)中的曲率,为远程形状感测带来了新的可能性。
\\
螺旋多芯光纤中的连续光栅 & ~2017 & 关于连续光栅多芯光纤的研究有了新的进展。
复合多芯光纤的商业可用性对于开发高分辨率和分布式FOSS来说至关重要。
\\
\bottomrule
\end{tabular}
\caption{FOSS的发展趋势}
\end{center}
\end{table}


\subsection{曲线重建算法}
\subsubsection{连续化算法}
\subsubsection{拟合算法}
\subsection{三维渲染技术}
\subsubsection{OpenGL}
\subsubsection{WebGL和Three.js}
\subsection{网络数据传输技术}
\subsubsection{HTTP}
\subsubsection{WebSocket}
\subsubsection{gRPC}
\subsubsection{TLS}
\section{研究展望}
\newpage
\printbibliography[title={参考文献}]
